\documentclass[11pt,a4paper,ngerman]{scrartcl}
 
\usepackage[utf8]{inputenc}
\usepackage[T1]{fontenc}
\usepackage{lmodern}
\usepackage[ngerman]{babel}
\usepackage{amsmath} 
\usepackage{amsfonts}
\usepackage{amssymb}
\usepackage{framed}

\usepackage{glossaries}
\title{Anmerkungen zum Paper}
\author{Horst Schneider}
\date{\today{}}

\newglossaryentry{geo-replicated}
{
  name=geo-replicated,
  description={bezeichnet die räumlich voneinander getrennte Verteilung von replizierten (nicht-relationalen) Datenbanken}
}

\newglossaryentry{eventual consistency}
{
  name=eventual consistency,
  description={bezeichnet die Einschränkung, dass Daten nach dem CAP-Theorem zwar sofort verfügbar und partition tolerant sind, dafür aber nicht sofort konsistent sind, sondern erst an einem unbestimmten Punkt in der Zukunft}
}

\makeglossaries

\begin{document}
 
\maketitle 
\pagebreak
\tableofcontents
\section{Abstract}
\begin{itemize}
\item \Gls{geo-replicated} Datenbanken garantieren sofortige Verfügbarkeit und Partitionstoleranz auf Kosten schwacher Konsistenz -> \gls{eventual consistency}
\item es gibt einige Verwirrung über die Semantik von \gls{eventual consistency}
\item die Vergleichbarkeit verschiedener Semantiken von \gls{eventual consistency} ist unmöglich, da die Semantik oft nur implizit durch die Implementierung gegeben ist bzw. keine Formalismen bei der Beschreibung verwendet werden
\end{itemize}

\begin{framed}
Ziel des papers ist es, ein Framework zur deklarativen Spezifikation von \gls{eventual consistency} zu bieten, das auf Axiomen basiert.
\end{framed}

\section{Introduction}


\printglossaries

\end{document}